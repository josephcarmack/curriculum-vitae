\documentclass[10pt]{article}
\usepackage[T1]{fontenc}
\usepackage{array, xcolor, lipsum, bibentry}
\usepackage{longtable}
\usepackage[margin=1in]{geometry}

\usepackage{titlesec}
\titleformat{\section}
  {\normalfont\Large\bfseries}{\thesection}{1em}{}[{\titlerule[0.8pt]}]

\title{\bfseries\Huge Joseph M. Carmack}
\date{}

\definecolor{lightgray}{gray}{0.8}
\newcolumntype{F}{>{\raggedright}p{0.35\textwidth}}
\newcolumntype{D}{p{0.65\textwidth}}
\newcolumntype{L}{>{\raggedleft}p{0.18\textwidth}}
\newcolumntype{K}{>{\raggedleft}b{0.18\textwidth}}
\newcolumntype{C}{>{\centering}p{0.18\textwidth}}
\newcolumntype{R}{p{0.76\textwidth}}
\newcolumntype{M}{p{0.76\textwidth}}
\newcommand\VRule{\color{lightgray}\vrule width 0.5pt}

\usepackage{hyperref}
\hypersetup{
    colorlinks=true,
    linkcolor=blue,
    filecolor=magenta,      
    urlcolor=blue,
    }

\begin{document}
\maketitle

\section*{Personal Info}
\begin{tabular}{F!{\VRule}D}
    {\bf Mailing Address}& 136 Amherst Street, Milford, NH 03055 \\
    {\bf Mobile Number}& (801) 755-2771 \\
    {\bf Email Address}& joseph.liping@gmail.com \\
    {\bf Personal Website}& \url{josephcarmack.github.io} \\
    {\bf Last Updated}& 08/18/2021
\end{tabular}

\section*{Values}

I value working in an intellectually rigorous and open environment where ideas
are exchanged freely with the intent of developing new technology to broadly
impact society for good. I highly esteem diversity and the refinement that comes from
innovating with a team from varied backgrounds towards a common goal.
Respecting differences and honesty are qualities I highly value in those I work
with and strive to embody myself. I know that a unified team with relentless
commitment to the scientific method and their mission can achieve what many
believe is impossible.

\section*{Research Interests}

Broadly I'm interested in understanding intelligence and how its implemented in
the human brain and machines. I've spent the last half-decade working in the
field of deep learning and am accutely aware of its strengths and weaknesses.
Although I've been working in deep learning professionally, I have also sought
a broader understanding of intelligence by studying biological-based theories
of intelligence and cognitive theories of the human mind. This more holistic
approach to studying intelligence has lead me to the belief that {\bf
intelligence can be defined as the efficiency at which an agent can convert
prior knowledge and experience into skill across large and varied task
distributions}. This clear and quantifiable definition has lead me to further
believe that many of the {\bf current gaps between deep learning and
human-levels of intelligence can be overcome by leveraging unsupervised
learning, identifying the inductive biases responsible for learning causal
models of an agents environment, and developing robust AI task curriculum.} I
believe all of these areas can be effectively studied and explored in the
domain of deep reinforcement learning. In addition to increasing the
intelligence of deep learning based agents, establishing methods to validate
what they have learned and not learned to build trust in the human consumers of
their services is crucial to accelerate the adoption of AI technologies. These
areas and problems largely define my research interests.

\section*{Education}
\begin{tabular}{L!{\VRule}R}
    2013-2018&{\bf PhD in Computational Material Science, {\it University of Arkansas}, GPA 4.0}\\[5pt]
    \vspace{2pt}&\vspace{2pt}\\
    2009--2013&{BS in Physics, {\it Brigham Young University--Idaho}, GPA 3.87}\\
\end{tabular}

\section*{Skills}
\begin{tabular}{F!{\VRule}D}
    {\bf Programming}& C++, Python, Matlab, Bash, Git, Make \\
    {\bf Machine Learning}& Neural Networks, Closed and OpenSet Classification, Dimensionality Reduction, Manifold Learning \\
    {\bf Reinforcement Learning}& Deep RL (model \& model-free), OpenAI Gym, RLLib \\
    {\bf High Performance and Cloud Computing}& CUDA, Docker, Linux Power User\\
    {\bf Data Analysis \& Visualization}& Paraview, Matplotlib, Bokeh, Plotly, Dashboards\\
    {\bf Data Science \& Algorithms}& Sci-kit Learn, PCA, UMAP, TSNE, Pandas, Dask, Rapids\\
    {\bf Math}& Probability, Statistics, Linear Algebra, Differential Equations, Vector Calculus, Graph Theory\\
    {\bf New Business Development}& Technical Concept Development, Proposal Writing, Customer Engagement\\
    {\bf Scientific Writing}& Literature reviews,  Journal \& Conference Article Preparation and Peer Review\\
\end{tabular}

\section*{Professional Experience}

\begin{longtable}{C!{\VRule}R}

    \textbf{2018--present}&{\bf Principal Scientist (Active TS-SCI
    Clearance) Cognitive RF Group} \\
    \textbf{BAE Systems FAST Labs\textsuperscript{TM}}&
    {
    \vspace{2pt}
    Lead development and execution of advanced Cognitive RF
    research programs, capture intellectuall property, and transition
    research into discriminating product capabilities. 
    }\\
    \vspace{2pt}&\vspace{2pt}\\
    &
    {
    \vspace{2pt}
    {\bf(2020-2021) Algorithm Team Lead} on the
    \href{https://beta.sam.gov/opp/a09982b4ddc54349a4845f106752ff50/view?keywords=SPiNN&sort=-relevance&index=&is_active=false&page=1&organization_id=300000412&date_filter_index=0&inactive_filter_values=false}{DARPA
    Signal Processing in Neural Networks Program}. Developed neural
    networks to replace standard digital signal processing (DSP) blocks
    in an LTE receiver using physics based inductive biases to increase
    architecture computational efficiency for deployment in edge IoT
    devices. Developed \textbf{Generative Adversarial Network (GAN)
    architectures} to model common RF impairments such as tone
    interference to be used to enhance the DSP based neural network
    training for increased robustness to edge cases. 
    }\\
    \vspace{2pt}&\vspace{2pt}\\
    &
    {
    \vspace{2pt}
    {\bf(2018-2020) Algorithm Team Lead} on the
    \href{https://www.darpa.mil/program/radio-frequency-machine-learning-systems}{DARPA
    RF Machine Learning Systems Program} where we {\bf established a
    new state-of-the-art for RF Fingerprint classification} by
    developing a highly sample efficient dilated causal convolutional
    neural network architecture and {\bf multi-agent deep reinforcement
    learning waveform synthesis} algorithm. 
    }\\
    \vspace{2pt}&\vspace{2pt}\\
    &
    {
    \vspace{2pt}
    {\bf(2021) Principal Investigator} on the Channel-Invariant RF
    Fingerprinting IRAD program. Developed extensions to RF
    Fingerprinting deep learning classifiers with inductive biases
    allowing them to be bline to RF multi-path propagation noise.
    }\\
    \vspace{2pt}&\vspace{2pt}\\
    &
    {
    \vspace{2pt}
    \textbf{(2020-2021)} Core developer for the DeepMission(TM) software framework for
    fast prototyping of deep reinforcement learning for defense
    applications with state-of-the-art algorithms. This framework
    brought together several open source packages including OpenAI's
    \href{https://gym.openai.com/}{Gym} framework,
    \href{https://spinningup.openai.com/en/latest/}{Spinning Up}
    library, Ray's
    \href{https://docs.ray.io/en/latest/rllib.html}{RLLib},
    \href{https://github.com/slundberg/shap}{SHAP}, along with custom
    in-house algorithms for instrospection into policy understanding
    and explainability.
    }\\
    \vspace{2pt}&\vspace{2pt}\\
    &
    {
    \vspace{2pt}
    \textbf{(2019) Principal Investigator} for the RF Environment Awareness
    with Explainable Results IRAD program. Developed explainable AI for
    RF Scene classification.
    }\\
    \vspace{2pt}&\vspace{2pt}\\
    &
    {
    \vspace{2pt}
    \textbf{(2018)} Algorithm Developer for the BAE System's team on the
    \href{https://www.darpa.mil/program/spectrum-collaboration-challenge}{DARPA
    Spectrum Collaboration Challenge} Program.
    }\\
    \vspace{2pt}&\vspace{2pt}\\

    \textbf{2013}&{\bf Materials Research, Nuclear Fuel Modeling \& Simulation
    Group} \\
    \textbf{Idaho National Laboratory}&
    {
    \vspace{2pt}
    Studied grain boundary properties in Uranium Dioxide nuclear fuel for the
    purpose of finding methods for optimizing fuel microstructure.  Activities
    included studying scientific literature, developing atomistic scale grain
    boundary models, running simulations to probe the UO2 properties using high
    performance computing resources, and communication of research at
    professional international research conference.
    }\\
    \vspace{2pt}&\vspace{2pt}\\

    \textbf{2012}&{\bf Physics Research, Condensed Matter Group} \\
    \textbf{Brigham Young University}&
    {
    \vspace{2pt}
    Characterized x-ray crystallography data to demonstrate novel theoretical
    framework for understanding complex magnetic structures and symmetry of
    materials at the atomic scale. Responsibilities included understanding and
    applying advanced solid-state physics, group theory, and linear algebra to
    experimental data. Learning and using sophisticated x-ray crystallography
    software on experimental data. Preparing presentations and written reports
    to communicate research findings and important results.
    }\\
    \vspace{2pt}&\vspace{2pt}\\

    \textbf{2006-2007}&{\bf Finish Carpenter} \\
    \textbf{Hayward Enterprises}&
    {
    \vspace{2pt}
    Built and installed custom hardwood furniture and cabinets. Operated high powered
    tools and shop equipment according to safety standards and procedures.
    }\\
\end{longtable}

\section*{Teaching Experience}
\begin{longtable}{C!{\VRule}R}
    \textbf{2016--2017}&{\bf Numerical Methods for Engineers Full Course Instructor}\\
    \textbf{University of Arkansas}&
    {
    \vspace{2pt}
    Average class of 50 undergraduate students, prepared and adapted lecture
    notes, delivered lectures, developed online learning videos to supplement
    lecture material (see my website), taught engineers to program in Matlab and Python
    and implement state of the art numerical methods using hands on homework
    assignments, prepared and administered exams to assess learning and assign
    grades, ran drill sessions and held office hours to mentor individual
    students. Performed all grading of homeworks and exams personally.
    }\\
    \vspace{2pt}&\vspace{2pt}\\
    \textbf{2013}&{\bf Adjunct Physics Faculty}\\
    \textbf{Brigham Young University Idaho}&
    {
    \vspace{2pt}
    Taught introductory physics course to undergraduates from diverse backgrounds and
    majors. Prepared and delivered lectures, assigned and graded homeworks, and held
    office hours. Prepared, administered, and graded exams.
    }\\
    \vspace{2pt}&\vspace{2pt}\\
    \textbf{2007--2010}&{\bf English and Mandarin Chinese Tutor}\\
    &
    {
    \vspace{2pt}
    Taught English as a second language for free as a volunteer in Taiwan (2007-2009).
    Classes were held weekly and anyone from the community was invited to attend. Class
    sizes from 5-30 with students from extremely varied backgrounds (small children to 
    elderly and beginner to advanced levels). Private Mandarin Chinese tutoring (2009-2010).
    Online English tutoring for high school students in Taiwan (2010).
    }\\
\end{longtable}

\bibliographystyle{plain}
\nobibliography{publications}

\section*{Peer Reviewed Publications }

\href{https://scholar.google.com/citations?user=dfwJUzsAAAAJ&hl=en&oi=ao}{Google Scholar}

\begin{longtable}{L!{\VRule}R}
    2021&\bibentry{carmack2021riftnet} {\bf(Best Paper Award!)}\\
    \vspace{1pt}&\vspace{1pt}\\
	    &\bibentry{carmack2021multi}\\
    \vspace{1pt}&\vspace{1pt}\\
	    &\bibentry{schmidt2021riftnext}\\
    \vspace{1pt}&\vspace{1pt}\\
        &\bibentry{kuzdeba2021transfer}\\
    \vspace{1pt}&\vspace{1pt}\\
    2020&\bibentry{robinson2020dilated}\\
    \vspace{1pt}&\vspace{1pt}\\
    2019&\bibentry{stankowicz2019complex}\\
    \vspace{1pt}&\vspace{1pt}\\
    2018&\bibentry{carmack2018tuning}\\
    \vspace{1pt}&\vspace{1pt}\\
    2017&\bibentry{carmack2017diverse} {\bf(Cover Article!)}\\
    \vspace{1pt}&\vspace{1pt}\\
    2015&\bibentry{carmack2015numerical}
\end{longtable}

\section*{Conference Presentations}
\begin{longtable}{L!{\VRule}R}
    2021&\bibentry{carmack2021riftnet} {\bf(Best Presentation Award!)}\\
    \vspace{1pt}&\vspace{1pt}\\
	    &\bibentry{carmack2021multi}\\
    \vspace{1pt}&\vspace{1pt}\\
    2020&\bibentry{stankowicz2019complex}\\
    \vspace{1pt}&\vspace{1pt}\\
    2018&\bibentry{carmack2018tuning}\\
    \vspace{1pt}&\vspace{1pt}\\
    2017&\bibentry{carmack2017diverse}\\
    \vspace{1pt}&\vspace{1pt}\\
    2015&\bibentry{carmack2015numerical}\\
    \vspace{1pt}&\vspace{1pt}\\
    2014&Carmack JM, Millett PC. Nanoparticle Self-assembly in Binary Polymer
    Blends at the Mesoscale in thin film geometries with wetting. {\it Material
    Research Society (MRS) Fall Meeting}. Boston, MA \\
    \vspace{1pt}&\vspace{1pt}\\
    2013&Carmack JM, Zhang Y, Bai X. Molecular dynamics simulations of grain
    boundary properties in UO2. {\it The Minerals, Metals, and Materials
    Society (TMS) Spring Meeting}. San Diego, California
\end{longtable}

\section*{Thesis and Dissertation}
\begin{tabular}{L!{\VRule}R}
    2018&{\bf PhD Dissertation:} \bibentry{carmack2018mesoscale}\\
    2013&{\bf Physics Senior Thesis:} \bibentry{carmack2013extending} 
\end{tabular}

\section*{Public Coding Projects}

\subsection{C++ Machine Learning Artificial Intelligence Toolkit}
\begin{description}
    \item[Repository:] www.github.com/josephcarmack/MLAI\texttt{\_}toolkit
\end{description}
This repository contains a library of C++ classes that implement several
machine learning and artificial intelligence algorithms. Although opensource libraries existed containing these same algorithms I used this repo to teach myself. I also found that my implmentations were faster than some of the standard open source packages since they were developed in C++ instead of Python at the time. Algorithm implementations include:
 \begin{itemize}
     \itemsep-0.5em
     \item Neural Networks (fully-connected layers), K-Nearest Neighbors, Logistic Regression
     \item Bayesian Belief Networks with Gibbs Sampling
     \item Deep Q-Learning
     \item Dimensionality Reduction Algorithms such as Principle Component
         Analysis, ISOMAP, and Maximum Variance Unfolding.
     \item Optimization Algorithms such as hill climbing, gradient descent,
         and stochastic gradient descent i.e. backpropagation.
 \end{itemize}

\subsection{Meso--A C++ High Performance Computing Phase-field Simulation Code}
\begin{description}
    \item[Repository:] \url{www.github.com/josephcarmack/meso/tree/develop}
\end{description}
I am a primary developer of the mesoscale simulation code for the Paul Millett
Research Group at the University of Arkansas. It is used for studying the
synthesis of soft materials such as multiphase polymer mixtures, mixtures with
colloidal particle, and block copolymer mixtures. It supports multiple physics
models such as diffusion based Cahn-Hilliard-Cook models, Lattice Boltzmann
Models, and Langavin particle dynamics. It supports multiple numerical solvers
such as finite difference solvers and fast pseudo-spectral solvers. It is a
massively parallel code meaning that it scales well to running on modern super
computer clusters using the Message Passing Interface (MPI) for distributing
computational work across multiple CPU nodes. Its development has been
supported by multiple grants from the National Science Foundation and other
government funding sources.

\subsection{Cueso--A CUDA C++ version of the Meso Simulation code (see above)}
\begin{description}
    \item[Repository:] \url{www.github.com/josephcarmack/cueso}
\end{description}
Primary creator and developer of this version of the Meso simulation code in
the Paul Millett research group. The purpose of this code is to add GPU acceleration
support to the physics models being explored in the Millett research group.

\subsection{Peso--A C++ Meso simulation code post processor}
\begin{description}
    \item[Repository:] \url{www.github.com/josephcarmack/peso/tree/develop}
\end{description}
A parallel processing set of post-processors for the Meso simulation code.

\subsection{Parstubuilder--A Python code for automating large parametric studies}
\begin{description}
    \item[Repository:] \url{www.github.com/josephcarmack/parstubuilder/tree/develop}
\end{description}
I designed and developed this code to automate directory structure generation,
batch Job submission, and batch job deletion of input file based HPC
applications. It is used in the Paul Millett Research Group to efficiently
conduct very large parametric studies with the Meso simulation code.

\subsection{A Simple Python Molecular Dynamics Code}
\begin{description}
    \item[Repository:] \url{www.github.com/josephcarmack/python\_molecular\_dynamics}
\end{description}
I wrote this code as a way to learn python and also prototype particle interactions
for my PhD research.

\section*{Languages}

\begin{tabular}{L!{\VRule}R}
    English&Mother tongue\\
    \vspace{2pt}&\vspace{2pt}\\
    Mandarin&Fluent (at one time)\\
\end{tabular}

\section*{Community Service}
\begin{itemize}
    \itemsep-0.5em
    \item {\bf Church Youth Counselor:} Mentor to church youth in grades K-12.
        Responsibilities have flunctuated over the years according to the needs
        of the local youth I have served but in general have included teaching
        them life skills, helping them plan and prepare for future education,
        ethics and leadership coaching, and activity planning and execution ,
        e.g. weekend camp outs, annual funderaisers, summer camps etc.
        \textbf{(2012 - present)}
    \item {\bf Assistant Scout Master:} volunteered as an assistant scout
        master in my local church sponsored Boy Scout's of America troop. I
        mentored individual scouts multiple times per week at troop meetings,
        on monthly camp outs, and during our annual BSA summer camp.
        \textbf{(2013-2015)}
    \item {\bf Cub Scout Bear Den Leader:} My wife and I volunteered to be Cub
        Scout leaders in our community. We held weekly den meetings with groups
        of 5-10 boys ages 8-9 and taught them life skills, took them on field
        trips, and helped them advance through the Cub Scout ranking system.
        \textbf{(2012-2013)}
    \item {\bf Volunteer Missionary Service:} I served as a missionary for my
        church, the {\it Church of Jesus Christ of Latter-day Saints}, in Taiwan
        for two years. As a missionary I lived a rigorous schedule
        (14-15 hour work days year round with no vacation) filled with community
        service, gospel study, and active proselyting. During my mission tenure
        I had many leadership opportunities such as being responsible for the training, well
        being, and success of other groups of missionaries from 10-40 in number and
        spread over large geographic regions. \textbf{(2007-2009)}
\end{itemize}

\section*{Awards and Honors}
\begin{itemize}
    \itemsep-0.5em
    \item Graduate Doctoral Academy Fellowship at the {\it University of Arkansas}
        (\$10k for 4 years 2013-2017)
    \item Eagle Scout (2004)
    \item {\it Brigham Young University Idaho} Academic Excellence Full Tuition Scholarship (2011-2013)
    \item {\it Sigma Pi Sigma} Physics Honor Society nomination (top 5\% of my class)
\end{itemize}

\section*{Interests}
Philosophy of science, physical fitness, homesteading, hobby electronics (Arduino),
intelligence (brain theory, sentience, perception, etc), classical education, music
(guitar and ukulele), space travel...

{
    % \vspace{20pt}\newline\newline
    \vspace{20pt}
    \scriptsize\hfill Created using \LaTeX
}

\end{document}

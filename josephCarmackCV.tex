\documentclass[10pt]{article}
\usepackage[T1]{fontenc}
\usepackage{array, xcolor, lipsum, bibentry}
\usepackage[margin=1in]{geometry}

\usepackage{titlesec}
\titleformat{\section}
  {\normalfont\Large\bfseries}{\thesection}{1em}{}[{\titlerule[0.8pt]}]

\title{\bfseries\Huge Joseph M Carmack}
\date{}

\definecolor{lightgray}{gray}{0.8}
\newcolumntype{F}{>{\raggedright}p{0.35\textwidth}}
\newcolumntype{D}{p{0.65\textwidth}}
\newcolumntype{L}{>{\raggedleft}p{0.14\textwidth}}
\newcolumntype{R}{p{0.8\textwidth}}
\newcommand\VRule{\color{lightgray}\vrule width 0.5pt}

\begin{document}
\maketitle

\section*{Personal Info}
\begin{tabular}{F!{\VRule}D}
    {\bf Mailing Address}& 109 Dakota Trail, Farmington, Arkansas 72730 \\
    {\bf Mobile Number}& (801) 755-2771 \\
    {\bf Email Address}& joseph.liping@gmail.com\\
    {\bf Personal Website}& josephcarmack.github.io\\
    {\bf Last Updated}& \today
\end{tabular}

\section*{Statement of Intent}

I desire to work in an intellectually open environment where ideas are
exchanged freely with the intent of developing new technology to broadly impact
society. I value diversity and the refinement that comes from innovating with a
team from varied backgrounds towards a common goal.  Respecting differences and
honesty are qualities I highly value in those I work with and strive to embody
myself. I believe that a unified team, with commitment and constant effort can
achieve amazing results that will bring about a bright future.

\section*{Research Interests}

I would like to assist in efforts to demystify intelligence and how it is
implemented in both biological and machine systems. My research has focused
on understanding existing cognitive architectures and their ability to generalise
to a broad spectrum of problems. I have spent time surveying and implementing
state of the art {\bf deep learning techniques for supervised, unsupervised, and
reinforcement learning} tasks. I am interested in new deep learing algorithms
and training techniques, as well as drawing inspiration from biologicial theories
of intelligence in order to create {\bf next generation machine learning and artificial
intelligence algorithms}.

\section*{Education}
\begin{tabular}{L!{\VRule}R}
    2013-present&{\bf PhD in Mechanical Engineering, University of Arkansas, GPA 4.0}\\[5pt]
    \vspace{2pt}&\vspace{2pt}\\
    2009--2013&{BS in Physics, Brigham Young University--Idaho, GPA 3.87}\\
\end{tabular}

\section*{Skills}
\begin{tabular}{F!{\VRule}D}
    {\bf Programming}& C++, Python, Matlab, Bash, Git \\
    {\bf Machine Learning}& Neural Networks, Classification, Dimensionality Reduction \\
    {\bf Artificial Intelligence}& Deep Reinforcement Learning, Belief Networks \\
    {\bf High Performance Computing}& CUDA C++, MPI, Linux\\
    {\bf Data Analysis \& Visualization}& Paraview, Python (Matplotlib, Numpy)\\
\end{tabular}

\section*{Professional Experience}
\begin{tabular}{L!{\VRule}R}
    2013--today&{\bf Research Assitant} in the Paul Millett Research Group,
    {\bf University of Arkansas}\\

    &
    {
    \vspace{2pt}
    Research in computational materials science. I Developed high performance
    computing software in C++ to run high fidelity mesoscale material simulations.
    Simulation models were used to study and understand complex synthesis processes
    for advanced membranes.
    }\\
    \vspace{2pt}&\vspace{2pt}\\
    summer 2016&{\bf Sales Representative for Shine Solar LLC}\\
    &
    {
    \vspace{2pt}
    Prospected my own leads, closed \$30k worth of product.
    }\\
    \vspace{2pt}&\vspace{2pt}\\
    Summer 2013&{\bf Research Intern} in the Nuclear Fuels Modeling and
    Simulation Research Group, {\bf Idaho National Laboratory}\\
    &
    {
    \vspace{2pt}
    Studied grain boundary properties in Uranium Dioxide nuclear fuel for the
    purpose of finding methods for optimizing fuel microstructure.
    Responsibilities included studying scientific literature, developing
    atomistic scale grain boundary models, and running simulations to probe the
    models properties using high perfromance computing resources.
    }\\
    \vspace{2pt}&\vspace{2pt}\\
    2006-2007&{\bf Finish Carpenter at Hayward Enterprizes}\\
    &
    {
    \vspace{2pt}
    Built and installed custom wood furniture and cabinets. Operated high powered
    tools and shop equipment according to safety standards and procedures.
    }\\
\end{tabular}

\section*{Teaching Experience}
\begin{tabular}{L!{\VRule}R}
    2016--2017&{\bf Numerical Methods for Engineers Full Course Instructor} in
    the Department of Mechanical Engineering at the {\bf University of
    Arkansas}\\
    &
    {
    \vspace{2pt}
    Average class of 50 undergraduate students, prepared and adapted lecture
    notes, delivered lectures, developed online learning videos to supplement
    lecture material (see my website), taught engineers to program in Matlab
    and implement state of the art numerical methods using hands on homework
    assignments, prepared and administered exams to asses learning and assign
    grades, ran drill sessions and held office hours to mentor individual
    students. Performed all grading of homeworks and exams personally.
    }\\
    \vspace{2pt}&\vspace{2pt}\\
    Summer 2013&{\bf Adjunct Faculty} in the Physics Department at{\bf Brigham
    Young University Idaho}\\
    &
    {
    \vspace{2pt}
    Taught introductory physics course to undergraduates from diverse backgrounds and
    majors. Prepared and delivered lectures, assigned and graded homeworks, and held
    office hours. Prepared, administered, and graded exams.
    }\\
    \vspace{2pt}&\vspace{2pt}\\
    2007--2010&{\bf English and Mandarin Chinese Tutor}\\
    &
    {
    \vspace{2pt}
    Taught English as a second langauge for free as a volunteer in Taiwan (2007-2009).
    Classes were held weekly and anyone from the community was invited to attend. Class
    sizes from 5-30 with students from extremely varied backgrounds (small children to 
    elderly and beginner to advanced levels). Private Mandarin Chinese tutoring (2009-2010).
    Online English tutoring for high school students in Taiwan (2010).
    }\\
\end{tabular}

\bibliographystyle{plain}
\nobibliography{publications}

\section*{Peer Reviewed Publications}
\begin{tabular}{L!{\VRule}R}
    2017&\bibentry{carmack2017diverse} {\bf(Cover Article!)}\\
    2015&\bibentry{carmack2015numerical}
\end{tabular}

\section*{Thesis and Dissertation}
\begin{tabular}{L!{\VRule}R}
    2013&\bibentry{carmack2013extending}{\bf  (Physics Senior Thesis)}
\end{tabular}

\section*{Projects}

\subsection{C++ Machine Learning Artificial Intelligence Toolkit}
\begin{description}
    \item[Repository:] www.github.com/josephcarmack/MLAI\texttt{\_}Toolkit
\end{description}
This repository contains a library of C++ classes that implement several
machine learning and artificial intelligence algorithms including:
\begin{itemize}
    \itemsep-0.5em
    \item Neural Networks, K-Nearest Neighbors, Logistic Regression
    \item Bayesian Belief Networks with Gibb's Sampling
    \item Deep Q-Learning
    \item Dimensionality Reduction Algorithms such as Principle Component
        Analysis, ISOMAP, and Maximum Variance Unfolding.
    \item Optimization Algorithms such as hill climbing, gradient descent,
        and stochastic gradient descent.
\end{itemize}

\subsection{Meso--A C++ High Performance Computing Phase-field Simulation Code}
\begin{description}
    \item[Repository:] www.github.com/josephcarmack/meso/tree/develop
\end{description}
I am a core developer of the mesoscale simulation code for the Paul Millett
Research Group at the University of Arkansas. It is used for studying the
synthesis of soft materials such as multiphase polymer mixtures, mixtures with
colloidal particle, and block copolymer mixtures. It supports multiple physics
models such as diffusion based Cahn-Hilliard-Cook models, Lattice Boltzmann
Models, and Langavin particle dynamics. It supports multiple numerical solvers
such as finite difference solvers and fast pseudo-spectral solvers. It is a
massively parallel code meaning that it scales well to running on modern super
computer clusters using the Message Passing Interface (MPI) for distributing
computational work across multiple CPU nodes. Its development has been
supported by multiple grants from the National Science Foundation and other
government funding sources.

\subsection{Peso--A C++ Meso simulation code post processor}
\begin{description}
    \item[Repository:] www.github.com/josephcarmack/peso/tree/develop
\end{description}

\subsection{Parstubuilder--A Python code for automating large parametric studies}
\begin{description}
    \item[Repository:] www.github.com/josephcarmack/parstubuilder/tree/develop
\end{description}
I designed and developed this code to automate directory structure generation,
batch Job submission, and batch job deletion of input file based HPC
applications. It is used in the Paul Millett Research Group to efficiently
conduct very large parametric studies with the Meso simulation code.

\subsection{A Simple Python Molecular Dynamics Code}
\begin{description}
    \item[Repository:] www.github.com/josephcarmack/python\_molecular\_dynamics
\end{description}
I wrote this code as a way to learn python and also prototype particle interactions
for my PhD research.

\section*{Languages}

\begin{tabular}{L!{\VRule}R}
    English&Mother tongue\\
    \vspace{2pt}&\vspace{2pt}\\
    Mandarin&Fluent (at one time)\\
\end{tabular}

\section*{Community Service}
\begin{itemize}
    \itemsep-0.5em
    \item {\bf Volunteer Missionary Service:} I served as a missionary for my
        church, the Church of Jesus Christ of Latter-day Saints, in Taiwan
        for two years. As a missionary I lived a rigorous schedule
        (14-15 hour work days year round with no vacation) filled with community
        service, gospel study, and active proselyting. During my mission tenure
        I had many leadership opportunities such as being responsible for the training, well
        being, and success of other groups of missionaries from 10-40 in number and
        spread over large geographic regions. (2007-2009)
    \item {\bf Cub Scout Bear Den Leader:} My wife and I volunteered to be Cub
        Scout leaders in our community. We held weekly den meetings with groups
        of 5-10 boys ages 8-9 and taught them life skills, took them on field
        trips, and helped them advance through the Cub Scout ranking system.
        (2012-2013)
    \item {\bf Assistant Scout Master:} I volunteered as an assistant scout
        master in my local church sponsored Boy Scout's of America troop. I
        mentored individual scouts multiple times per week at troop meetings,
        on monthly campouts, and during our annual BSA summer camp. (2013-2015)
\end{itemize}

\section*{Awards and Honors}
\begin{itemize}
    \itemsep-0.5em
    \item Eagle Scout (2004)
    \item Brigham Young University Idaho Academic Excellence Full Tuition Scholarship (2011-2013)
    \item \$2,000 Society of Physics Students Scholarship (2012)
    \item Sigma Pi Sigma Physics Honor Society nomination (top 5\% of my class)
    \item 3rd place for best oral presentation at BYUI undergraduate research conference
    \item Graduate Doctoral Academy Fellowship at the University of Arkansas
        (\$10k for 4 years 2013-2017)
    \item 1st place for best poster in University of Arkansas poster competition (2015)
\end{itemize}

\section*{Interests}
Philosophy of science, physical fitness, homesteading, hobby electronics (Arduino),
intelligence (brain theory, sentience, perception, etc), classical education, music
(piano and ukulele), space travel...

{
    % \vspace{20pt}\newline\newline
    \vspace{20pt}
    \scriptsize\hfill Created using \LaTeX
}

\end{document}
